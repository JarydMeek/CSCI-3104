
%%%%%%%%%%%%%%%%%%%%%%%%%%%%%%%%%%%%%%%%%%%%%%%%%%%%%%%%%%
%%
%% This is the "PREAMBLE". Here we define the type of document and load in any packages we might want. You can also set parameters %% % and create your own short-hand here.
%%
%%%%%%%%%%%%%%%%%%%%%%%%%%%%%%%%%%%%%%%%%%%%%%%%%%%%%%%%%%

 \documentclass[9pt]{article}
 
 \def\solutions{1}


 \usepackage{amsmath}
 \usepackage{amssymb}
 \usepackage{graphicx}    % needed for including graphics e.g. EPS, PS \usepackage{tikz}
 \usepackage{tikz}
 \usepackage{tikzsymbols}
 \usepackage{relsize}
 \usetikzlibrary{patterns,decorations.pathreplacing,shapes,arrows}
 \usepackage{algorithm2e}
 \topmargin -2.5cm        % read Lamport p.163
 \oddsidemargin -0.04cm   % read Lamport p.163
 \evensidemargin -0.04cm  % same as oddsidemargin but for left-hand pages
 \textwidth 16.59cm
 \textheight 25.94cm
% \pagestyle{empty}        % Uncomment if don't want page numbers
 \pagenumbering{gobble}
 \parskip 7.2pt           % sets spacing between paragraphs
 %\renewcommand{\baselinestretch}{1.5} 	% Uncomment for 1.5 spacing between lines
 \parindent 0pt		  % sets leading space for paragraphs

 \usepackage{listings}% http://ctan.org/pkg/listings
\lstset{
  basicstyle=\ttfamily,
  mathescape
}
% No date in header
\date{}

\usepackage{hyperref}
\hypersetup{
    colorlinks=true,
    linkcolor=blue,
    filecolor=magenta,      
    urlcolor=cyan,
}
\usepackage{amsthm}
\usepackage{fancyhdr}
\pagestyle{fancy}
\setlength{\headsep}{36pt}

\usepackage{hyperref}

\newcommand{\lp}{\left(}
\newcommand{\rp}{\right)}
\newcommand{\lb}{\left[}
\newcommand{\rb}{\right]}
\newcommand{\ls}{\left\{}
\newcommand{\rs}{\right\}}
\newcommand{\lbar}{\left|}
\newcommand{\rbar}{\right|}
\newcommand{\ld}{\left.}
\newcommand{\rd}{\right.}

\newcommand{\myexists}{\exists \hspace{.3mm}}

\newcommand{\hs}{\hspace{.75mm}}
\newcommand{\bs}{\hspace{-.75mm}}
\newcommand{\nin}{\noindent}

\newcommand{\fx}{f\bs\left( x \right)}
\newcommand{\gx}{g\bs\left( x \right)}
\newcommand{\qx}{q\bs\left( x \right)}

\newcommand{\nn}{\nonumber}

\newcommand{\vfive}{\vspace{5mm}}
\newcommand{\vthree}{\vspace{3mm}}

\newcommand{\fof}[1]{f\lp #1\rp}
\newcommand{\gof}[1]{g\lp #1\rp}
\newcommand{\qof}[1]{q\lp #1\rp}

\newcommand{\myp}[1]{\left( #1 \right)}
\newcommand{\myb}[1]{\left[ #1 \right]}
\newcommand{\mys}[1]{\left\{ #1 \right\}}
\newcommand{\myab}[1]{\left| #1 \right|}

\newcommand{\myj}{_j}
\newcommand{\myjp}{_{j+1}}
\newcommand{\myjm}{_{j-1}}

\newcommand{\f}[1]{f\hspace{-1mm}\left( #1 \right)}
\newcommand{\fp}[1]{f'\hspace{-1mm}\left( #1 \right)}
\newcommand{\g}[1]{g\hspace{-1mm}\left( #1 \right)}
\newcommand{\gp}[1]{g'\hspace{-1mm}\left( #1 \right)}
\newcommand{\q}[1]{q\hspace{-1mm}\left( #1 \right)}
\newcommand{\qp}[1]{q'\hspace{-1mm}\left( #1 \right)}
\newcommand{\Px}[1]{P\hspace{-1mm}\left( x_{#1} \right)}
\newcommand{\Qx}[1]{Q\hspace{-1mm}\left( x_{#1} \right)}

\newcommand{\tten}[1]{\times 10^{#1}}

\newcommand{\aij}[1]{a_{#1}}
\newcommand{\bij}[1]{b_{#1}}
\newcommand{\rij}[1]{r_{#1}}

\newcommand{\R}[1]{\mathbb{R}^{#1}}

\newcommand{\ith}{i^{\textrm{th}}}
\newcommand{\jth}{i^{\textrm{th}}}
\newcommand{\kth}{i^{\textrm{th}}}

\newcommand{\inv}[1]{{#1}^{-1}}

\newcommand{\bx}{\mathbf{x}}
\newcommand{\bv}{\mathbf{v}}
\newcommand{\bw}{\mathbf{w}}
\newcommand{\by}{\mathbf{y}}
\newcommand{\bb}{\mathbf{b}}
\newcommand{\be}{\mathbf{e}}
\newcommand{\br}{\mathbf{r}}
\newcommand{\xhat}{\hat{\mathbf{x}}}

\newcommand{\beq}{\begin{eqnarray}}
\newcommand{\eeq}{\end{eqnarray}}

\newcommand{\ben}{\begin{enumerate}}
\newcommand{\een}{\end{enumerate}}

\newcommand{\bsq}{\mathsmaller{\blacksquare}}

\newcommand{\iter}[1]{^{\myp{#1}}}

% matrix macro
\newcommand{\mymat}[1]{
\left[
\begin{array}{rrrrrrrrrrrrrrrrrrrrrrrrrrrrrrrrrrrrrrr}
#1
\end{array}
\right]
}

\newcommand{\makenonemptybox}[2]{%
%\par\nobreak\vspace{\ht\strutbox}\noindent
\item[]
\fbox{% added -2\fboxrule to specified width to avoid overfull hboxes
% and removed the -2\fboxsep from height specification (image not updated)
% because in MWE 2cm is should be height of contents excluding sep and frame
\parbox[c][#1][t]{\dimexpr\linewidth-2\fboxsep-2\fboxrule}{
  \hrule width \hsize height 0pt
  #2
 }%
}%
\par\vspace{\ht\strutbox}
}
\makeatother

\newcommand{\smallaug}[1]{
\left[
\begin{array}{rr|r}
#1
\end{array}
\right]
}

%%%%%%%%%%%%%%%%%%%%%%%%%%%%%%%%%%%%%%%%%%%%%%%%%%%%%%%%%%
%%
%% End of PREAMBLE
%%
%%%%%%%%%%%%%%%%%%%%%%%%%%%%%%%%%%%%%%%%%%%%%%%%%%%%%%%%%%



% ======================================================================================
% Actual document starts here. 
% PLEASE FILL IN YOUR NAME AND STUDENT ID.
% ======================================================================================
\begin{document}

\lhead{{\bf CSCI 3104, Algorithms \\ Problem Set 3 (50 points)} }
\rhead{Name: \fbox{Jaryd Meek}  \\  ID: \fbox{} \\ Collaborators: Noah Nguyen \\ {\bf Due February 5, 2021 \\ Spring 2021, CU-Boulder}}
\renewcommand{\headrulewidth}{0.5pt}

\phantom{Test}

\begin{small}
\textit{Advice 1}:\ For every problem in this class, you must justify your answer:\ show how you arrived at it and why it is correct. If there are assumptions you need to make along the way, state those clearly.
\vspace{-3mm} 

\textit{Advice 2}:\ Verbal reasoning is typically insufficient for full credit. Instead, write a logical argument, in the style of a mathematical proof.\\
\vspace{-3mm} 

\textbf{Instructions for submitting your solution}:
\vspace{-5mm} 

\begin{itemize}
	\item The solutions \textbf{should be typed} and we cannot accept hand-written solutions. \href{http://ece.uprm.edu/~caceros/latex/introduction.pdf}{Here's a short intro to Latex.}
	\item You should submit your work through \href{https://www.gradescope.com/courses/218966}{\textbf{Gradescope}} only.
	\item The easiest way to access Gradescope is through our Canvas page. There is a Gradescope button in the left menu.
	\item Gradescope will only accept \textbf{.pdf} files.
	\item \href{https://www.youtube.com/watch?v=u-pK4GzpId0&feature=emb_logo}{It is vital that you match each problem part with your work.} Skip to 1:40 to just see the matching info.
\end{itemize}
\vspace{-4mm} 
\end{small}

\hrulefill
\pagebreak



\ben
%%%%%%%%%%%%%%%%%%%%%%%%%%%%%%%%%%%%%%%%%%%%%%%%%%%%%%%%
% PROBLEM  ONE %% PROBLEM  ONE %% PROBLEM  ONE %% PROBLEM  ONE %% PROBLEM  ONE %
%==============================================================================
% Problem 1: Worst Case Analysis
%==============================================================================
% PROBLEM  ONE %% PROBLEM  ONE %% PROBLEM  ONE %% PROBLEM  ONE %% PROBLEM  ONE %
%%%%%%%%%%%%%%%%%%%%%%%%%%%%%%%%%%%%%%%%%%%%%%%%%%%%%%%%

\item Name (a) one advantage, (b) one disadvantage, and (c) one alternative to worst-case analysis. For (a) and (b) you should use full sentences.

  \if\solutions1
  \vspace{2mm}
  
  \textbf{Solution:}   \\
%==============================================================================
% STUDENTS: TYPE YOUR SOLUTIONS HERE. (Between \textbf{Solution:} and \fi )
%==============================================================================
\ben
	\item \textit{Advantage - } One advantage of the worst-case analysis is the fact that no matter what you know the absolute slowest that your algorithm or function will take. 
	\\
	\item \textit{Disadvantage - } One disadvanage of the worst-case analysis is the fact that it shows the worst-case, not the average time. While it may be nice to know the absolute longest it may take your alogrithm to run, it may be far more useful to know the average time it will take to run.
	\\
	\item \textit{Alternative - } One alternative to the worst-case analysis is the average-case analysis.
\een

\fi

\newpage


%%%%%%%%%%%%%%%%%%%%%%%%%%%%%%%%%%%%%%%%%%%%%%%%%%%%%%%%
% PROBLEM TWO %% PROBLEM TWO %% PROBLEM TWO %% PROBLEM TWO %% PROBLEM TWO %
%==============================================================================
% Problem 2: Growth Rates
%==============================================================================
% PROBLEM TWO %% PROBLEM TWO %% PROBLEM TWO %% PROBLEM TWO %% PROBLEM TWO %
%%%%%%%%%%%%%%%%%%%%%%%%%%%%%%%%%%%%%%%%%%%%%%%%%%%%%%%%


\vspace{5mm}

\item 
Put the growth rates in order, from slowest-growing to fastest. That is, if your answer is $f_1(n), f_2(n), \dotsc, f_k(n)$, then $f_i(n) \leq O(f_{i+1}(n))$ for all $i$. If two adjacent ones have the same order of growth (that is, $f_i(n) = \Theta(f_{i+1}(n))$), you must specify this as well. Justify your answer (show your work). 
\begin{itemize}
\item You may assume transitivity: if $f(n) \leq O(g(n))$ and $g(n) \leq O(h(n))$, then $f(n) \leq O(h(n))$, and similarly for little-oh, etc. Note that the goal is to order the growth rates, so transitivity is very helpful. We encourage you to make use of transitivity rather than comparing all possible pairs of functions, as using transitivity will make your life easier.

\item You may also use the limit test (see Michael's Calculus Notes on Canvas, referred to as the Limit Comparison Test). However, you \textbf{MUST} show all limit computations at the same level of detail as in Calculus I-II.
\item You may \textbf{NOT} use heuristic arguments, such as comparing degrees of polynomials or identifying the ``high order term" in the function.
\item If it is the case that $g(n) = c \cdot f(n)$ for some constant $c$, you may conclude that $f(n) = \Theta(g(n))$ without using Calculus tools. You must clearly identify the constant $c$ (with any supporting work necessary to identify the constant- such as exponent or logarithm rules) and include a sentence to justify your reasoning. 
\end{itemize}


\begin{enumerate}
\item \label{2a} {\itshape.
\[
2^n, \qquad 
n^2, \qquad
\frac{1}{n}, \qquad
1, \qquad
6n+10^{100}, \qquad
\frac{1}{\sqrt{n}}, \qquad
\log_5(n^2), \qquad
\log_2(n), \qquad
3^n.
\]
}

\end{enumerate}

\if\solutions1
\vspace{2mm}

\textbf{Solution:} \\
%==============================================================================
% STUDENTS: TYPE YOUR SOLUTIONS HERE. (Between \textbf{Solution:} and \fi )
%==============================================================================
Begin limit comparison tests to see which functions grow fastest.\\

Comparing $2^n$ vs $n^2$ - \\
$ \displaystyle \lim_{x \to \infty} \frac{2^n}{n^2} \stackrel{L'H}{=} \lim_{x \to \infty} \frac{n\cdot 2^{n-1}}{2n} \stackrel{L'H}{=} \lim_{x \to \infty} \frac{n(n-1)\cdot 2^{n-2}}{2} \stackrel{\text{Evaluate at }\infty}{\rightarrow} \frac{\infty}{2} = \infty$ \hfill \textbf{Therefore $\mathbf{n^2}$ is $\mathbf{o(2^n)}$}\\
\\Comparing $n^2$ vs $\frac{1}{n}$ - \\
$ \displaystyle \lim_{x \to \infty} \frac{n^2}{\frac{1}{n}} \stackrel{L'H}{=} \lim_{x \to \infty} \frac{2n}{-\frac{1}{n^2}} \stackrel{L'H}{=} \lim_{x \to \infty} \frac{2}{-\frac{2}{n^3}} \stackrel{\text{Evaluate at }\infty}{\rightarrow} \frac{2}{-\frac{2}{\infty}} = \frac{2}{0} = \infty$ \hfill \textbf{Therefore $\mathbf{\frac{1}{n}}$ is $\mathbf{o(n^2)}$}\\

$$\text{\textbf{Therefore we have proven the order thus far is -}} $$
$$\mathbf{\frac{1}{n},\quad n^2,\quad 2^n}$$

Comparing $\frac{1}{n}$ vs $1$ - \\
$ \displaystyle \lim_{x \to \infty} \frac{\frac{1}{n}}{1}  \stackrel{\text{Evaluate at }\infty}{\rightarrow} \frac{\frac{1}{\infty}}{1} = \frac{0}{1} = \infty$ \hfill \textbf{Therefore $\mathbf{\frac{1}{n}}$ is $\mathbf{o(1)}$}\\
Comparing $1$ vs $n^2$ - \\
$ \displaystyle \lim_{x \to \infty} \frac{1}{n^2}  \stackrel{\text{Evaluate at }\infty}{\rightarrow} \frac{1}{\infty} = \frac{0}{\infty} = 0$ \hfill \textbf{Therefore $\mathbf{1}$ is $\mathbf{o(n^2)}$}\\

$$\text{\textbf{Therefore we have proven the order thus far is -}} $$
$$\mathbf{\frac{1}{n},\quad 1 ,\quad n^2,\quad 2^n}$$

\newpage

Comparing $1$ vs $6n+10^{100}$ - \\
$ \displaystyle \lim_{x \to \infty} \frac{1}{6n+10^{100}} \stackrel{\text{Evaluate at }\infty}{\rightarrow} \frac{1}{\infty} = 0$ \hfill \textbf{Therefore $\mathbf{1}$ is $\mathbf{o(6n+10^{100})}$}\\
\\Comparing $n^2$ vs $6n+10^{100}$ - \\
$ \displaystyle \lim_{x \to \infty} \frac{n^2}{6n+10^{100}} \stackrel{L'H}{=} \lim_{x \to \infty} \frac{2n}{6} \stackrel{\text{Evaluate at }\infty}{\rightarrow} \frac{\infty}{6} = \infty$ \hfill \textbf{Therefore $\mathbf{6n+10^{100}}$ is $\mathbf{o(n^2)}$}\\

$$\text{\textbf{Therefore we have proven the order thus far is -}} $$
$$\mathbf{\frac{1}{n},\quad 1 ,\quad 6n+10^{100},\quad n^2,\quad 2^n}$$

Comparing $6n+10^{100}$ vs $\frac{1}{\sqrt n}$ - \\
$ \displaystyle \lim_{x \to \infty} \frac{6n+10^{100}}{\frac{1}{\sqrt n}} \stackrel{L'H}{=} \frac{6}{-\frac{1}{2 \sqrt[3]{n}}} \stackrel{\text{Evaluate at }\infty}{\rightarrow} \frac{6}{-\frac{1}{\infty}} =  \frac{6}{0} = \infty$ \hfill \textbf{Therefore $\mathbf{\frac{1}{\sqrt n}}$ is $\mathbf{o(6n+10^{100})}$}\\
Comparing $1$ vs $\frac{1}{\sqrt n}$ - \\
$ \displaystyle \lim_{x \to \infty} \frac{1}{\frac{1}{\sqrt n}} \stackrel{\text{Evaluate at }\infty}{\rightarrow} \frac{1}{\frac{1}{\infty}} =  \frac{1}{0} = \infty$ \hfill \textbf{Therefore $\mathbf{\frac{1}{\sqrt n}}$ is $\mathbf{o(1)}$}\\
Comparing $\frac{1}{n}$ vs $\frac{1}{\sqrt n}$ - \\
$ \displaystyle \lim_{x \to \infty} \frac{\frac{1}{n}}{\frac{1}{\sqrt n}} \stackrel{\text{Multiply by }n}{\rightarrow} \lim_{x \to \infty} \frac{1}{\frac{n}{\sqrt n}} = \frac{1}{n \cdot n^{-\frac{1}{2}}} = \frac{1}{\sqrt{n}} \stackrel{\text{Evaluate at }\infty}{\rightarrow} \frac{1}{\infty} = 0$ \hfill \textbf{Therefore $\mathbf{\frac{1}{n}}$ is $\mathbf{o(\frac{1}{\sqrt n})}$}\\

$$\text{\textbf{Therefore we have proven the order thus far is -}} $$
$$\mathbf{\frac{1}{n},\quad \frac{1}{\sqrt n} ,\quad 1 ,\quad 6n+10^{100},\quad n^2,\quad 2^n}$$

Comparing $\frac{1}{\sqrt n}$ vs $\log_5 (n^2)$ - \\
$ \displaystyle \lim_{x \to \infty} \frac{\frac{1}{\sqrt n}}{\log_5 (n^2)} \stackrel{\text{Evaluate at }\infty}{\rightarrow} \frac{\frac{1}{\infty}}{\infty} =  \frac{0}{\infty} = 0$ \hfill \textbf{Therefore $\mathbf{\frac{1}{\sqrt n}}$ is $\mathbf{o(\log_5 (n^2))}$}\\
Comparing $1$ vs $\log_5 (n^2)$ - \\
$ \displaystyle \lim_{x \to \infty} \frac{1}{\log_5 (n^2)} \stackrel{\text{Evaluate at }\infty}{\rightarrow} \frac{1}{\infty} = 0$ \hfill \textbf{Therefore $\mathbf{1}$ is $\mathbf{o(\log_5 (n^2))}$}\\
Comparing $6n+10^{100}$ vs $\log_5 (n^2)$ - \\
$ \displaystyle \lim_{x \to \infty} \frac{6n+10^{100}}{\log_5 (n^2)} = \frac{6}{\frac{2 \ln n}{\ln 5}} \stackrel{L'H}{=} \lim_{x \to \infty} \frac{6}{\frac{2}{n\ln 5}} \stackrel{\text{Evaluate at }\infty}{\rightarrow} \frac{6}{\frac{2}{\infty}} = \frac{6}{0} = \infty$ \; \textbf{Therefore $\mathbf{\log_5 (n^2)}$ is $\mathbf{o(6n+10^{100})}$}\\

$$\text{\textbf{Therefore we have proven the order thus far is -}} $$
$$\mathbf{\frac{1}{n},\quad \frac{1}{\sqrt n} ,\quad 1,\quad \log_5 (n^2) ,\quad 6n+10^{100},\quad n^2,\quad 2^n}$$

Comparing $\log_5 (n^2)$ vs $\log_2 (n)$ - \\
$ \displaystyle \lim_{x \to \infty} \frac{\log_5 (n^2)}{\log_2 (n)} = \frac{\frac{2 \ln n}{\ln 5}}{\frac{\ln n}{\ln 2}} \stackrel{L'H}{=} \lim_{x \to \infty} \frac{\frac{2}{n\ln 5}}{\frac{1}{n\ln 2}} \stackrel{\text{Simplify}}{\rightarrow} \lim_{x \to \infty} \frac{2 \cdot n \cdot \ln 2}{1 \cdot n \cdot \ln 5} = \lim_{x \to \infty} \frac{2 \cdot \ln 2}{1 \cdot \ln 5} \stackrel{\text{Evaluate at }\infty}{\rightarrow} = \frac{2 \cdot \ln 2}{\ln 5} = c$\\
\begin{flushright}
\textbf{Therefore since the LCT yielded a constant, $\mathbf{\log_5 (n^2)}$ is $\mathbf{\Theta(\log_2 (n))}$}\\
\textbf{Contant = $\frac{2 \ln 2}{\ln 5}$}\\
\end{flushright}

$$\text{\textbf{Therefore we have proven the order thus far is -}} $$
$$\mathbf{\frac{1}{n},\quad \frac{1}{\sqrt n} ,\quad 1,\quad \lp\; \log_5 (n^2)\quad \log_2 (n) \;\rp,\quad 6n+10^{100},\quad n^2,\quad 2^n}$$

\newpage

Comparing $\log_2 (n)$ vs $3^n$ - \\
$ \displaystyle \lim_{x \to \infty} \frac{\log_2 (n)}{3^n} \stackrel{L'H}{=} \lim_{x \to \infty} \frac{\frac{1}{n \ln 2}}{n \cdot 3^{n-1}} \stackrel{\text{Evaluate at }\infty}{\rightarrow}  = \frac{\frac{1}{\infty}}{\infty} = \frac{0}{\infty} = 0$ \hfill \textbf{Therefore $\mathbf{\log_2 (n)}$ is $\mathbf{o(3^n)}$}\\
Comparing $6n+10^{100}$ vs $3^n$ - \\
$ \displaystyle \lim_{x \to \infty} \frac{6n+10^{100}}{3^n} \stackrel{L'H}{=} \lim_{x \to \infty} \frac{6}{n \cdot 3^{n-1}} \stackrel{\text{Evaluate at }\infty}{\rightarrow}  = \frac{6}{\infty} = 0$ \hfill \textbf{Therefore $\mathbf{6n+10^{100}}$ is $\mathbf{o(3^n)}$}\\
Comparing $n^2$ vs $3^n$ - \\
$ \displaystyle \lim_{x \to \infty} \frac{n^2}{3^n} \stackrel{L'H}{=} \lim_{x \to \infty} \frac{2n}{n \cdot 3^{n-1}} \stackrel{L'H}{=} \lim_{x \to \infty} \frac{2}{n(n-1) \cdot 3^{n-2}} \stackrel{\text{Evaluate at }\infty}{\rightarrow}  = \frac{2}{\infty} = 0$ \hfill \textbf{Therefore $\mathbf{n^2}$ is $\mathbf{o(3^n)}$}\\
Comparing $2^n$ vs $3^n$ - \\
$ \displaystyle \lim_{x \to \infty} \frac{2^n}{3^n} \stackrel{\text{Simplify}}{\rightarrow} \lim_{x \to \infty} \lp \frac{2}{3}\rp^n \stackrel{\text{Evaluate at }\infty}{\rightarrow}  = \lp \frac{2}{3} \rp ^\infty = 0$ \hfill \textbf{Therefore $\mathbf{n^2}$ is $\mathbf{o(3^n)}$}\\


$$\text{\textbf{Therefore we have proven the order is -}} $$
\begin{align}
\boxed{$$\mathbf{\frac{1}{n},\quad \frac{1}{\sqrt n} ,\quad 1,\quad \lp\; \log_5 (n^2)\quad \log_2 (n) \;\rp,\quad 6n+10^{100},\quad n^2,\quad 2^n, \quad 3^n}$$} \nonumber
\end{align}
\fi
\newpage


%%%%%%%%%%%%%%%%%%%%%%%%%%%%%%%%%%%%%%%%%%%%%%%%%%%%%%%%
% PROBLEM THREE %% PROBLEM THREE %% PROBLEM THREE %% PROBLEM THREE %% PROBLEM THREE %
%==============================================================================
% Problem 3: Worst Case Running Time
%==============================================================================
% PROBLEM THREE %% PROBLEM THREE %% PROBLEM THREE %% PROBLEM THREE %% PROBLEM THREE %
%%%%%%%%%%%%%%%%%%%%%%%%%%%%%%%%%%%%%%%%%%%%%%%%%%%%%%%%

\vspace{5mm}

\item 
Analyze the worst-case running time for each of the following algorithms. You should give your answer in Big-Theta notation. You \emph{do not} need to give an input which achieves your worst-case bound, but you should try to give as tight a bound as possible. 
	\begin{itemize}
	\item In the column to the right of the code, indicate the cost of executing each line once.
	
	\item In the next column (all the way to the right), indicate the number of times each line is executed.
	
	\item Below the code, justify your answers (show your work), and compute the total runtime in terms of Big-Theta notation.  You may assume that $T(n)$ is the Big-Theta of the high-order term. You need not use Calculus techniques. However, you \textbf{must} show the calculations to determine the closed-form expression for a summation (represented with the $\displaystyle \sum$ symbol). You cannot simply say: $\displaystyle \sum_{i=1}^{n} i \in \Theta(n^{2})$, for instance.
	
	\item In both columns, you don't have to put the \emph{exact} values. For example, putting ``$c$'' for constant is fine. We will not be picky about off-by-one errors (i.e., the difference between $n$ and $n-1$); however, we will be picky about off-by-two errors (i.e., the difference between $n$ and $n-2$).
    \end{itemize}


\begin{enumerate}
\item Consider the following algorithm. \label{3a}
\begin{lstlisting}
                                               | Cost     | # times run
1  f(A[1, ..., n]): 
2    ans = 0                                   | $C_1$         | $1$
3    for i = 1 to n-1:                         | $C_2$         | $n$
4      for j = i+1 to n:                       | $C_3$         | $(\sum_{i=1}^{n} t_i) -1$
5        if A[i] > A[j]:                       | $C_4$         | $\sum_{i=1}^{n-1} t_i$
6           ans += 1                           | $C_5$         | $\sum_{i=1}^{n-1} t_i$
7
8    return ans                                | $C_6$         | $1$
\end{lstlisting}

\textbf{Justification - }\\
	\textit{line 2 -}  The line only runs once when the variable is initialized.\\
	\\\textit{line 3 -}  The line runs $n-1$ times when it enters the loop, and one more time when the loop terminates, therefore running $n$ total times.\\
	\\\textit{line 4 -}  This line runs n times the first time (n-1 loops, and 1 termination), n-1 times the second time (n-2 loops and 1 termination), and so on. Therefore this line runs $(n-1) + (n-2) + ... + 3 + 2 + 1 =  \sum_{i=1}^{n} t_i$ times, except that  on the last iteration itdoesn't run a termination step since the outer loop terminates first, therefore the number of times this line runs is $(\sum_{i=1}^{n} t_i) -1$\\
	\\\textit{line 5 -}  This line runs n-1 times the first time, n-2 times the second time, and so on until it runs 1 time. Therefore this line runs $(n-1) + (n-2) + (n-3) ... + 3 + 2 + 1 = \sum_{i=1}^{n-1} t_i$\\
	\\\textit{line 6 -}  This line runs a maximum of n-1 times the first time, n-2 times the second time, and so on until it runs 1 time. Therefore this line runs $(n-1) + (n-2) + (n-3) ... + 3 + 2 + 1 = \sum_{i=1}^{n-1} t_i$\\
	\\\textit{line 8 -}  The line only runs once when returning.\\
	$$\text{\textbf{Solving T(n) is on the following page}}$$
\newpage

\textbf{Solving $T(n)$ -}\\
	$$t_i = i \text{ for every } i$$
	$$(\sum_{i=1}^{n} t_i) - 1 = (\sum_{i=1}^{n} i) - 1 = \frac{n(n+1)}{2} - 1$$
	$$\text{and}$$
	$$\sum_{i=1}^{n-1} t_i = \sum_{i=1}^{n-1} i = \frac{(n-1)n}{2}$$
	$$\text{Substitute the above in and solve}$$
	$$T(n) = C_1 \cdot 1 + C_2 \cdot n + C_3 \cdot \frac{n(n+1)}{2} - 1 + C_4 \cdot \frac{(n-1)n}{2} + C_5 \cdot \frac{(n-1)n}{2} + C_6 \cdot 1$$
	$$ = C_1 + C_2 \cdot n + C_3 \cdot \frac{n^2}{2} + C_3 \cdot \frac{n}{2} - C_3 + C_4 \cdot \frac{n^2}{2} - C_4 \cdot \frac{n}{2} + C_5 \cdot \frac{n^2}{2} - C_5 \cdot \frac{n}{2} + C_6$$
	$$ = \lp\frac{C_3}{2} + \frac{C_4}{2} + \frac{C_5}{2} \rp n^2 +  \lp C_2 + \frac{C_3}{2} - \frac{C_4}{2} - \frac{C_5}{2} \rp n + (C_1 + C_6 - C_3)$$
	$$ T(n) =  a n^2 + b n + c$$
	
	$$\text{Since } an^2 + b n + c \leq T(n) \leq (a + b + c) n^2$$
	\begin{align}
		\boxed{$$ T(n) \text{ is } \Theta (n^2)$$} \nonumber
	\end{align}
	\newpage

\item Write an algorithm to multiply two matrices $A$ with dimension $n \times m$ and $B$ with dimension $m \times n$. Write your algorithm in pseudocode in Latex, similar in style to the given pseudocode in $3(a)$. Analyze the worst-case runtime of your algorithm as above. You may use the subroutine \verb zeros([a,b]) ~to create an array of zeros with dimension $a \times b$. \verb zeros([a,b]) ~has runtime $\Theta(ab)$. \label{3a}

\end{enumerate}

\if\solutions1
\vspace{2mm}

\textbf{Solution:} \\
%==============================================================================
% STUDENTS: TYPE YOUR SOLUTIONS HERE. (Between \textbf{Solution:} and \fi )
%==============================================================================

\begin{lstlisting}
	                                       | Cost     | # times run
1 mutiplyMatrices(A, B, n, m): 
2    ans = zeros([n, n])                       | $C_1$         | $1$
3    for x = 1 to n:                           | $C_2$         | $n+1$
4     	for y = 1 to n:                        | $C_3$         | $n \cdot (n+1)$
5      	    cellSum = 0                        | $C_4$         | $n \cdot n$
6           for z = 1 to m:                    | $C_5$         | $n \cdot n \cdot (m+1)$
7               cellSum += (A[i,k] * B[k,j])   | $C_6$         | $n \cdot n \cdot m$
8           ans[i,j] = cellSum                 | $C_7$         | $n \cdot n$
9    return ans                                | $C_8$         | $1$
\end{lstlisting}

\textbf{Justification - }\\
	\textit{line 2 -}  The line only runs once when the variable is initialized.\\
	\\\textit{line 3 -} The line runs $n$ times when it enters the loop, and one more time when the loop terminates, therefore running $n+1$ total times.\\
	\\\textit{line 4 -} The line runs $n$ times from the outer for loop, and $n$ times for it's own loop, as well as 1 time when the loop terminates, so a total of $n(n+1)$ times.\\
	\\\textit{line 5 -} The line runs $n$ times for the outer loop and $n$ times for the inner loop, so $n\cdot n $ times total.\\
	\\\textit{line 6 -} The line runs $n$ times for the outer loop and $n$ times for the inner loop, as well as $m$ times for the third for loop, but runs one more time when the loop terminates. Therefore running a total of $n \cdot n \cdot (m+1)$ times. \\
	\\\textit{line 7 -} The line runs $n$ times for the outer loop and $n$ times for the inner loop, and $m$ times for the third loop, therefore running $n \cdot n \cdot m$ total times.\\
	\\\textit{line 8 -} The line runs $n$ times for the outer loop and $n$ times for the inner loop, so $n\cdot n $ times total.\\
	\\\textit{line 9 -} The line only runs once when returning.\\


	$$\text{\textbf{Solving T(n) is on the following page}}$$
\newpage

\textbf{Solving $T(n)$ -}\\
	$$T(n) = C_1 (1) + C_2 (n+1) + C_3 (n)(n+1) + C_4 (n)(n) + C_5 (n)(n)(m+1) + C_6 (n)(n)(m) + C_7 (n)(n) + C_8 (1)$$
	$$ = C_1 + C_2 (n)+ C_2 + C_3 (n^2) + C_3 (n) + C_4 (n^2) + C_5 (n^2 m) + C_5 (n^2) + C_6 (n^2 m) + C_7 (n^2) + C_8$$
	$$ \text{Before we continue simplifying, we should sub out } C_1 \text{ with the given runtime of zeros() which is }$$
	$$ \Theta (ab) \text { where } a = b = n \text{ so zeros() is } \Theta (n^2)$$
	$$\text{Therefore, we will represent } C_1 \text{ as } C_1 (n^2)$$
	$$ = C_1 (n^2) + C_2 (n)+ C_2 + C_3 (n^2) + C_3 (n) + C_4 (n^2) + C_5 (n^2 m) + C_5 (n^2) + C_6 (n^2 m) + C_7 (n^2) + C_8$$
	$$ = (C_5 + C_6)n^2m + (C_1 + C_3 + C_4 + C_5 + C_7)n^2 + (C_2 + C_3)n + (C_2 + C_8)$$
	$$ T(n) =  a n^2m + b n^2 + c n  + d$$
	$$ T(n) =  a n^2 + b n + c$$
	$$\text{Since } an^m \leq T(n) \leq (a + b + c + d)n^2m$$
	\begin{align}
		\boxed{$$ T(n) \text{ is } \Theta (n^2m)$$} \nonumber
	\end {align} 
\fi

\newpage

%==============================================================================
% Problem 4: Recurrence Relations
%==============================================================================
\vspace{5mm}

\item For the following algorithms, write down the recurrence relation associated with the runtime of the algorithm. \textbf{You do not} have to solve the recurrence relation.

\begin{enumerate}
    \item Algorithm A solves problems by dividing them into five sub-problems of half the size, recursively solving each sub-problem, and then combining the solutions in linear time. For a problem of size 1, the algorithm takes constant time.

    \item Algorithm B solves problems of size $n$ by recursively solving two sub-problems of size $n - 1$ and then combining the solutions in constant time. For a problem of size 1, the algorithm takes constant time.
    
    \item 
    	\begin{verbatim}
	def hello(n) {
	     if (n > 1) {
	        print( `hello' `hello' `hello' )
	        hello(n/7)
	        hello(n/7)
	        hello(n/7)
	        hello(n/7)
	}}
	\end{verbatim}
\end{enumerate}

\if\solutions1
\vspace{3mm}
{\bf Solution}: \\
%==============================================================================
% STUDENTS: TYPE YOUR SOLUTIONS HERE. (Between \textbf{Solution:} and \fi )
%==============================================================================

\ben
	\item 
	$ \begin{cases} 
		\Theta (1) & n = 1\\
		5T(n/2) + \Theta (n) & n > 1 
		
	 \end{cases} $
	\item 
	$ \begin{cases} 
		\Theta (1) & n = 1\\
		2T(n-1) + \Theta (1) & n > 1 
		
	 \end{cases} $
	\item
	$ \begin{cases} 
		\Theta (1) & n = 1\\
		4T(n/7) + \Theta (n) & n > 1 
		
	 \end{cases} $
\een

\fi


%========================================================================================================================

\een 


\end{document}
