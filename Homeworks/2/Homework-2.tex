
%%%%%%%%%%%%%%%%%%%%%%%%%%%%%%%%%%%%%%%%%%%%%%%%%%%%%%%%%%
%%
%% This is the "PREAMBLE". Here we define the type of document and load in any packages we might want. You can also set parameters %% % and create your own short-hand here.
%%
%%%%%%%%%%%%%%%%%%%%%%%%%%%%%%%%%%%%%%%%%%%%%%%%%%%%%%%%%%

 \documentclass[9pt]{article}
 
 \def\solutions{1}


 \usepackage{amsmath}
 \usepackage{amssymb}
 \usepackage{graphicx}    % needed for including graphics e.g. EPS, PS \usepackage{tikz}
 \usepackage{tikz}
 \usepackage{tikzsymbols}
 \usepackage{relsize}
 \usetikzlibrary{patterns,decorations.pathreplacing,shapes,arrows}
 \usepackage{algorithm2e}
 \topmargin -2.5cm        % read Lamport p.163
 \oddsidemargin -0.04cm   % read Lamport p.163
 \evensidemargin -0.04cm  % same as oddsidemargin but for left-hand pages
 \textwidth 16.59cm
 \textheight 25.94cm
% \pagestyle{empty}        % Uncomment if don't want page numbers
 \pagenumbering{gobble}
 \parskip 7.2pt           % sets spacing between paragraphs
 %\renewcommand{\baselinestretch}{1.5} 	% Uncomment for 1.5 spacing between lines
 \parindent 0pt		  % sets leading space for paragraphs

% No date in header
\date{}

\usepackage{hyperref}
\hypersetup{
    colorlinks=true,
    linkcolor=blue,
    filecolor=magenta,      
    urlcolor=cyan,
}
\usepackage{amsthm}
\usepackage{fancyhdr}
\pagestyle{fancy}
\setlength{\headsep}{36pt}

\usepackage{hyperref}

\newcommand{\lp}{\left(}
\newcommand{\rp}{\right)}
\newcommand{\lb}{\left[}
\newcommand{\rb}{\right]}
\newcommand{\ls}{\left\{}
\newcommand{\rs}{\right\}}
\newcommand{\lbar}{\left|}
\newcommand{\rbar}{\right|}
\newcommand{\ld}{\left.}
\newcommand{\rd}{\right.}

\newcommand{\myexists}{\exists \hspace{.3mm}}

\newcommand{\hs}{\hspace{.75mm}}
\newcommand{\bs}{\hspace{-.75mm}}
\newcommand{\nin}{\noindent}

\newcommand{\fx}{f\bs\left( x \right)}
\newcommand{\gx}{g\bs\left( x \right)}
\newcommand{\qx}{q\bs\left( x \right)}

\newcommand{\nn}{\nonumber}

\newcommand{\vfive}{\vspace{5mm}}
\newcommand{\vthree}{\vspace{3mm}}

\newcommand{\fof}[1]{f\lp #1\rp}
\newcommand{\gof}[1]{g\lp #1\rp}
\newcommand{\qof}[1]{q\lp #1\rp}

\newcommand{\myp}[1]{\left( #1 \right)}
\newcommand{\myb}[1]{\left[ #1 \right]}
\newcommand{\mys}[1]{\left\{ #1 \right\}}
\newcommand{\myab}[1]{\left| #1 \right|}

\newcommand{\myj}{_j}
\newcommand{\myjp}{_{j+1}}
\newcommand{\myjm}{_{j-1}}

\newcommand{\f}[1]{f\hspace{-1mm}\left( #1 \right)}
\newcommand{\fp}[1]{f'\hspace{-1mm}\left( #1 \right)}
\newcommand{\g}[1]{g\hspace{-1mm}\left( #1 \right)}
\newcommand{\gp}[1]{g'\hspace{-1mm}\left( #1 \right)}
\newcommand{\q}[1]{q\hspace{-1mm}\left( #1 \right)}
\newcommand{\qp}[1]{q'\hspace{-1mm}\left( #1 \right)}
\newcommand{\Px}[1]{P\hspace{-1mm}\left( x_{#1} \right)}
\newcommand{\Qx}[1]{Q\hspace{-1mm}\left( x_{#1} \right)}

\newcommand{\tten}[1]{\times 10^{#1}}

\newcommand{\aij}[1]{a_{#1}}
\newcommand{\bij}[1]{b_{#1}}
\newcommand{\rij}[1]{r_{#1}}

\newcommand{\R}[1]{\mathbb{R}^{#1}}

\newcommand{\ith}{i^{\textrm{th}}}
\newcommand{\jth}{i^{\textrm{th}}}
\newcommand{\kth}{i^{\textrm{th}}}

\newcommand{\inv}[1]{{#1}^{-1}}

\newcommand{\bx}{\mathbf{x}}
\newcommand{\bv}{\mathbf{v}}
\newcommand{\bw}{\mathbf{w}}
\newcommand{\by}{\mathbf{y}}
\newcommand{\bb}{\mathbf{b}}
\newcommand{\be}{\mathbf{e}}
\newcommand{\br}{\mathbf{r}}
\newcommand{\xhat}{\hat{\mathbf{x}}}

\newcommand{\beq}{\begin{eqnarray}}
\newcommand{\eeq}{\end{eqnarray}}

\newcommand{\ben}{\begin{enumerate}}
\newcommand{\een}{\end{enumerate}}

\newcommand{\bsq}{\mathsmaller{\blacksquare}}

\newcommand{\iter}[1]{^{\myp{#1}}}

% matrix macro
\newcommand{\mymat}[1]{
\left[
\begin{array}{rrrrrrrrrrrrrrrrrrrrrrrrrrrrrrrrrrrrrrr}
#1
\end{array}
\right]
}

\newcommand{\makenonemptybox}[2]{%
%\par\nobreak\vspace{\ht\strutbox}\noindent
\item[]
\fbox{% added -2\fboxrule to specified width to avoid overfull hboxes
% and removed the -2\fboxsep from height specification (image not updated)
% because in MWE 2cm is should be height of contents excluding sep and frame
\parbox[c][#1][t]{\dimexpr\linewidth-2\fboxsep-2\fboxrule}{
  \hrule width \hsize height 0pt
  #2
 }%
}%
\par\vspace{\ht\strutbox}
}
\makeatother

\newcommand{\smallaug}[1]{
\left[
\begin{array}{rr|r}
#1
\end{array}
\right]
}

%%%%%%%%%%%%%%%%%%%%%%%%%%%%%%%%%%%%%%%%%%%%%%%%%%%%%%%%%%
%%
%% End of PREAMBLE
%%
%%%%%%%%%%%%%%%%%%%%%%%%%%%%%%%%%%%%%%%%%%%%%%%%%%%%%%%%%%



% ======================================================================================
% Actual document starts here. 
% PLEASE FILL IN YOUR NAME AND STUDENT ID.
% ======================================================================================
\begin{document}

\lhead{{\bf CSCI 3104, Algorithms \\ Problem Set 2 (50 points)} }
\rhead{Name: \fbox{Jaryd Meek}  \\  ID: \fbox{} \\ Collaborators: Noah Nguyen \\ {\bf Due January 29, 2021 \\ Spring 2021, CU-Boulder}}
\renewcommand{\headrulewidth}{0.5pt}

\phantom{Test}

\begin{small}
\textit{Advice 1}:\ For every problem in this class, you must justify your answer:\ show how you arrived at it and why it is correct. If there are assumptions you need to make along the way, state those clearly.
\vspace{-3mm} 

\textit{Advice 2}:\ Verbal reasoning is typically insufficient for full credit. Instead, write a logical argument, in the style of a mathematical proof.\\
\vspace{-3mm} 

\textbf{Instructions for submitting your solution}:
\vspace{-5mm} 

\begin{itemize}
	\item The solutions \textbf{should be typed} and we cannot accept hand-written solutions. \href{http://ece.uprm.edu/~caceros/latex/introduction.pdf}{Here's a short intro to Latex.}
	\item You should submit your work through \href{https://www.gradescope.com/courses/218966}{\textbf{Gradescope}} only.
	\item The easiest way to access Gradescope is through our Canvas page. There is a Gradescope button in the left menu.
	\item Gradescope will only accept \textbf{.pdf} files.
	\item \href{https://www.youtube.com/watch?v=u-pK4GzpId0&feature=emb_logo}{It is vital that you match each problem part with your work.} Skip to 1:40 to just see the matching info.
\end{itemize}
\vspace{-4mm} 
\end{small}

\hrulefill
\pagebreak



\ben
%%%%%%%%%%%%%%%%%%%%%%%%%%%%%%%%%%%%%%%%%%%%%%%%%%%%%%%%
% PROBLEM  ONE %% PROBLEM  ONE %% PROBLEM  ONE %% PROBLEM  ONE %% PROBLEM  ONE %
%==============================================================================
% Problem 1: Logarithms & Exponents Review
%==============================================================================
% PROBLEM  ONE %% PROBLEM  ONE %% PROBLEM  ONE %% PROBLEM  ONE %% PROBLEM  ONE %
%%%%%%%%%%%%%%%%%%%%%%%%%%%%%%%%%%%%%%%%%%%%%%%%%%%%%%%%

\item The following problems are a review of logarithm and exponent topics.

\begin{enumerate}

\item Solve for $x$.

	\begin{enumerate}
	\item $3^{2x} =  81 $
	\item $3(5^{x-1}) = 375 $
	\item $ \log_3 x^2  = 4  $
	\end{enumerate}


\item Solve for $x$.

	\begin{enumerate}
	\item $x^2 - x = \log_5 25 $
	\item $ \log_{10} (x+3) - \log_{10} x = 1 $
	\end{enumerate}

\item Answer each of the following with a TRUE or FALSE. 

	\begin{enumerate}
	\item $ a^{\log_a x} = x $
	\item $ a^{\log_b x} = x $
	\item $ a = b^{\log_b a} $
	\item $\log_a x = \frac{\log_b x}{\log_b a} $
	\item $\log b^m = m \log b $
	\end{enumerate}
	
\end{enumerate}

  \if\solutions1
  \vspace{2mm}
  
  \textbf{Solution:}   \\
%==============================================================================
% STUDENTS: TYPE YOUR SOLUTIONS HERE. (Between \textbf{Solution:} and \fi )
%==============================================================================
\begin {enumerate}
	\item 
		\begin {enumerate}
			\item 
				$3x^2 = 81$\\
				$\log_3(81) = 2x $\hfill(Apply logarithm rules)\\
				$4 = 2x$\hfill(Simplify)\\
				\fbox{$\mathbf{2 = x}$}\hfill(Divide both sides by 2)\\
			\item 
				$3(5^{x-1}) = 375 $\\
				$5^{x-1} = 125 $\hfill(Divide both sides by 3)\\
				$\log_5(125) = x-1 $\hfill(Apply logarithm rules)\\
				$3 = x-1 $\hfill(Simplify)\\
				\fbox{$\mathbf{4 = x }$}\hfill(Add 1 to both sides)\\
			\item 
				$ \log_3 x^2  = 4 $ \\
				$3^{\log_3(x^2)} = 3^4 $\hfill(Raise each side as the exponent to 3)\\
				$x^2 = 3^4 $\hfill(Simplify)\\
				$x^2 = 81 $\hfill(Simplify)\\
				$\sqrt{x^2} = \sqrt{81} $\hfill(Take the square root of both sides)\\
				\fbox{$\mathbf{x = \pm 9 }$}\hfill(Simplify)\\
		\end{enumerate}
	\item 
		\begin {enumerate}
			\item 
				$x^2 - x = \log_5 25 $\\
				$x^2 - x = 2 $ \hfill (Simplify)\\
				$x( x-1 ) = 2$ \hfill (Factor)\\
				\fbox{$\mathbf{x = 2, -1}$} \hfill (Solve for roots)\\
			\item 
				$ \log_{10} (x+3) - \log_{10} x = 1 $ \\
				$ \log_{10} \left(\frac{x+3}{x}\right) = 1$ \hfill (Apply log rules (subtraction)) \\
				$ 10^{\log_{10} \left(\frac{x+3}{x}\right)} = 10^1 $ \hfill(Raise each side as the exponent to 10)\\
				$  \frac{x+3}{x} = 10$ \hfill (Simplify)\\
				$  x + 3 = 10x $ \hfill (Multiply both sides by $x$)\\
				$ 3 = 10x - x $ \hfill (Subtract $x$ from both sides)\\
				$ 3 = 9x $ \hfill (Simplify)\\ 
				\fbox{$ \mathbf{\frac{1}{3} = x} $} \hfill (Divide both sides by 9)
		\end{enumerate}
		\newpage
	\item 
		\begin {enumerate}
			\item $ a^{\log_a x} = x $ - \fbox{\textbf{TRUE}}
			\item $ a^{\log_b x} = x $ - \fbox{\textbf{FALSE}}
			\item $ a = b^{\log_b a} $ - \fbox{\textbf{TRUE}}
			\item $\log_a x = \frac{\log_b x}{\log_b a} $ - \fbox{\textbf{TRUE}}
			\item $\log b^m = m \log b $ - \fbox{\textbf{TRUE}}
		\end{enumerate}
\end{enumerate}

\fi

\newpage


%%%%%%%%%%%%%%%%%%%%%%%%%%%%%%%%%%%%%%%%%%%%%%%%%%%%%%%%
% PROBLEM TWO %% PROBLEM TWO %% PROBLEM TWO %% PROBLEM TWO %% PROBLEM TWO %
%==============================================================================
% Problem 2: Limits at Infinity Review
%==============================================================================
% PROBLEM TWO %% PROBLEM TWO %% PROBLEM TWO %% PROBLEM TWO %% PROBLEM TWO %
%%%%%%%%%%%%%%%%%%%%%%%%%%%%%%%%%%%%%%%%%%%%%%%%%%%%%%%%


\vspace{5mm}

\item Compute the following limits at infinity. Show all work and justify your answer.

\begin{enumerate}
\item $ \displaystyle \lim_{x \to \infty} \frac{3x^3 + 2}{9x^3 - 2x^2 +7} $
\item $ \displaystyle \lim_{x \to \infty} \frac{x^3}{e^{x/2}} $
\item $ \displaystyle \lim_{x \to \infty} \frac{\ln x^4}{x^3} $
\end{enumerate}

\if\solutions1
\vspace{2mm}

\textbf{Solution:} \\
%==============================================================================
% STUDENTS: TYPE YOUR SOLUTIONS HERE. (Between \textbf{Solution:} and \fi )
%==============================================================================

\begin{enumerate}
	\item 
		$ \displaystyle \lim_{x \to \infty} \frac{3x^3 + 2}{9x^3 - 2x^2 +7}$ \\
		$ \displaystyle \lim_{x \to \infty} \frac{3 + \frac{2}{x^3}}{9 - \frac{2}{x} + \frac{7}{x^3}} $ \hfill (Divide by $x^3$)\\
		$ \displaystyle \frac{3 + 0}{9 - 0 + 0 } = \frac{3}{9} = $ \fbox{$\mathbf{\frac{1}{3}}$} \hfill (Evaluate at infinity and simplify)
	\item 
		$ \displaystyle \lim_{x \to \infty} \frac{x^3}{e^{x/2}} $ \\
		$ \displaystyle \stackrel{L'H}{=} \lim_{x \to \infty} \frac{3x^2}{\frac{e^{x/2}}{2}} =  \lim_{x \to \infty} \frac{6x^2}{e^{x/2}}$ \hfill (Apply L'Hôpital's and simplify)\\ 
		$ \displaystyle \stackrel{L'H}{=} \lim_{x \to \infty} \frac{12x}{\frac{e^{x/2}}{2}}  =  \lim_{x \to \infty} \frac{24x}{e^{x/2}}$ \hfill (Apply L'Hôpital's and simplify)\\
		$ \displaystyle \stackrel{L'H}{=} \lim_{x \to \infty} \frac{24}{\frac{e^{x/2}}{2}}  =  \lim_{x \to \infty} \frac{48}{e^{x/2}}$ \hfill (Apply L'Hôpital's and simplify)\\
		$ \displaystyle \frac{48}{\infty} = $ \fbox{$\mathbf{0}$}\hfill (Evaluate at infinity and simplify)
	\item
		$ \displaystyle \lim_{x \to \infty} \frac{\ln x^4}{x^3} $ \\
		$ \displaystyle \lim_{x \to \infty} \frac{4\ln x}{x^3} $ \hfill (Apply logarithm rules)\\
		$ \displaystyle \stackrel{L'H}{=} \lim_{x \to \infty} \frac{4\frac{1}{x}}{3x^2} $ \hfill (Apply L'Hôpital's)\\
		$ \displaystyle \stackrel{L'H}{=} \lim_{x \to \infty} \frac{4}{3x^3} $ \hfill (Simplify)\\
		$ \displaystyle \frac{4}{\infty} = $ \fbox{$\mathbf{0}$}\hfill (Evaluate at infinity and simplify)
\end {enumerate}

\fi
\newpage


%%%%%%%%%%%%%%%%%%%%%%%%%%%%%%%%%%%%%%%%%%%%%%%%%%%%%%%%
% PROBLEM THREE %% PROBLEM THREE %% PROBLEM THREE %% PROBLEM THREE %% PROBLEM THREE %
%==============================================================================
% Problem 3: Limits at Infinity Review
%==============================================================================
% PROBLEM THREE %% PROBLEM THREE %% PROBLEM THREE %% PROBLEM THREE %% PROBLEM THREE %
%%%%%%%%%%%%%%%%%%%%%%%%%%%%%%%%%%%%%%%%%%%%%%%%%%%%%%%%

\vspace{5mm}

\item Compute the following limits at infinity.  Show all work and justify your answer.

\begin{enumerate}
\item For real numbers $m, n > 0$ compute $ \displaystyle \lim_{x \to \infty} \frac{x^m}{e^{nx}} $
\item What does this tell us about the rate at which $e^{nx}$ approaches infinity relative to $x^m$? A brief explanation is fine for this part.
\item For real numbers $m, n > 0$ compute $ \displaystyle  \lim_{x \to  \infty} \frac{(\ln  x)^n}{x^m} $
\item What does this tell us about the rate at which $(\ln  x)^n$ approaches infinity relative to $x^m$? A brief explanation is fine for this part.
\end{enumerate}
\if\solutions1
\vspace{2mm}

\textbf{Solution:} \\
%==============================================================================
% STUDENTS: TYPE YOUR SOLUTIONS HERE. (Between \textbf{Solution:} and \fi )
%==============================================================================

\begin {enumerate}
	\item  
		$\displaystyle \lim_{x \to \infty} \frac{x^m}{e^{nx}}$ \\
		$ \displaystyle \stackrel{L'H}{=} \lim_{x \to \infty} \frac{mx^{m-1}}{ne^{nx}} $ \hfill (Apply L'Hôpital's and simplify) \\
		$ \displaystyle \stackrel{L'H}{=} \lim_{x \to \infty} \frac{m(m-1)x^{m-2}}{n^2e^{nx}} $ \hfill (Apply L'Hôpital's and simplify) \\
		\\We see a pattern forming. If we repeat the pattern, we will eventually get \fbox{$ \mathbf{\displaystyle \lim_{x \to \infty} \frac{m!}{n^me^{nx}} = 0} $} for real numbers $m, n > 0$
	\item This shows that $x^m$ approaches infinity slower than $e^{nx}$. 
	\item 
		$ \displaystyle  \lim_{x \to  \infty} \frac{(\ln  x)^n}{x^m} $ \\
		\\$ \displaystyle  \lim_{x \to  \infty} \frac{\ln^n  x}{x^m} $  \hfill (Apply logarithm rules) \\
		\\$ \displaystyle  \stackrel{L'H}{=} \lim_{x \to  \infty} \frac{\frac{n\ln^{n-1}x}{x}}{mx^{m-1}} =  \lim_{x \to  \infty} \frac{n\ln^{n-1}x}{mx^{m}}$ \hfill (Apply L'Hôpital's and simplify)\\
		\\$ \displaystyle  \stackrel{L'H}{=} \lim_{x \to  \infty} \frac{\frac{n(n-1)\ln^{n-2}x}{x}}{m^2x^{m-1}} =  \lim_{x \to  \infty} \frac{n(n-1)\ln^{n-2}x}{m^2x^{m}}$ \hfill (Apply L'Hôpital's and simplify)\\
		\\We see a pattern forming. If we repeat the pattern, we will eventually get \fbox{$ \mathbf{\displaystyle \lim_{x \to \infty} \frac{n!}{m^nx^m} = 0} $} for real numbers $m, n > 0$
	\item This shows that $(\ln x)^n$ approaches infinity slowerr than $ x^m$.
\end {enumerate}

\fi

\newpage

%==============================================================================
% Problem 4: Root and Ratio Test Review
%==============================================================================
\vspace{5mm}

\item The problems in this question deal with the Root and the Ratio Tests. Determine the convergence or divergence of the following series. State which test you used.

\begin{enumerate}
\item  $ \displaystyle \sum_{n=1}^{\infty} \frac{e^{2n}}{n^n} $
\item $ \displaystyle \sum_{n=0}^{\infty} \frac{2^n}{n!} $
\item $ \displaystyle \sum_{n=0}^{\infty} \frac{n^2  2^{n+1}}{3^n}  $
\item $ \displaystyle \sum_{n=1}^{\infty} \left ( \frac{\ln n}{n} \right )^n $
\end{enumerate}

\if\solutions1
\vspace{3mm}
{\bf Solution}: \\
%==============================================================================
% STUDENTS: TYPE YOUR SOLUTIONS HERE. (Between \textbf{Solution:} and \fi )
%==============================================================================
\begin {enumerate} 
	\item \textbf{Root Test} \\
		$ \displaystyle \sum_{n=1}^{\infty} \frac{e^{2n}}{n^n} \rightarrow  \lim_{n \to  \infty} \left|\frac{e^{2n}}{n^n}\right|^{\frac{1}{n}}$ \hfill (Apply root test) \\
		$ \displaystyle \lim_{n \to  \infty} \left|\left(\frac{e^{2}}{n}\right)^n\right|^{\frac{1}{n}}$ \hfill (Factor) \\
		$ \displaystyle \lim_{n \to  \infty} \left|\frac{e^{2}}{n}\right|$ \hfill (Simplify) \\
		$ \displaystyle \left|\frac{e^{2}}{\infty}\right| = 0 $ \hfill (Evaluate at infinity and simplify)\\
		$ 0 < 1 $ Therefore by the root test, the series is \fbox{\textbf{Convergent}}\\
	\item \textbf{Ratio Test} \\
		$ \displaystyle \sum_{n=0}^{\infty} \frac{2^n}{n!} \rightarrow \lim_{n \to  \infty} \frac{\frac{2^{n+1}}{(n+1)!}}{\frac{2^n}{n!}}$ \hfill (Apply ratio test). \\
		\\$\displaystyle \lim_{n \to  \infty} \frac{n! \cdot 2^{n+1}}{2^n \cdot (n+1)!}$ \hfill (Simplify) \\
		\\$\displaystyle \lim_{n \to  \infty} \frac{n! \cdot 2^n \cdot 2}{2^n \cdot n! \cdot (n+1)}$ (Expand to allow for cancellation) \\
		\\$\displaystyle \lim_{n \to  \infty} \frac{2}{(n+1)}$ \hfill (Cancel like terms) \\
		$ \displaystyle \frac{2}{\infty} = 0 $ \hfill (Evaluate at infinity and simplify)\\
		$ 0 < 1 $ Therefore by the ratio test, the series is \fbox{\textbf{Convergent}}\\
	\newpage
	\item \textbf{Ratio Test} \\
		$ \displaystyle \sum_{n=0}^{\infty} \frac{n^2  2^{n+1}}{3^n} \rightarrow \frac{\frac{(n+1)^2\cdot 2^{n+1}}{3^{n+1}}}{\frac{n^2 \cdot 2^n}{3^n}}$ \hfill (Apply ratio test). \\
		\\$\displaystyle \lim_{n \to  \infty} \frac{3^n \cdot (n+1)^2 \cdot 2^ {n+1}}{3^{n+1} \cdot n^2 \cdot 2^n}$ \hfill (Simplify) \\
		\\$\displaystyle \lim_{n \to  \infty} \frac{3^n \cdot (n^2 + 2n +1) \cdot 2^n \cdot 2}{3^n \cdot 3 \cdot n^2 \cdot 2^n}$ \hfill (Expand to allow for cancellation) \\
		\\$\displaystyle \lim_{n \to  \infty} \frac{2(n^2 + 2n +1)}{ 3n^2}$ \hfill (Cancel like terms) \\
		\\$\displaystyle \lim_{n \to  \infty} \frac{2n^2 + 4n +2}{ 3n^2}$ \hfill (Distribute) \\
		\\$\displaystyle \lim_{n \to  \infty} \frac{2 + \frac{4}{n} + \frac{2}{n^2}}{ 3}$ \hfill (Divide by $n^2$) \\
		\\$\displaystyle \frac{2 + 0 + 0}{ 3}  = \frac{2}{3}$ \hfill (Evaluate at infinity and simplify)\\
		$ \frac{2}{3}< 1 $ Therefore by the ratio test, the series is \fbox{\textbf{Convergent}}\\

	\item \textbf{Root Test} \\
		$ \displaystyle \sum_{n=1}^{\infty} \left ( \frac{\ln n}{n} \right )^n \rightarrow \lim_{n \to  \infty} \left|\left(\frac{\ln n}{n}\right)^n\right|^{\frac{1}{n}}$ \hfill (Apply root test) \\
		\\ $ \displaystyle \lim_{n \to  \infty} \left|\frac{\ln n}{n}\right|$ \hfill (Simplify) \\
		\\ $\stackrel{L'H}{=} \displaystyle \lim_{n \to  \infty} \left|\frac{\frac{1}{n}}{1}\right| =  \lim_{n \to  \infty} \left|\frac{1}{n}\right|$ \hfill (Apply L'Hôpital's and simplify) \\
		\\$\displaystyle \left|\frac{1}{\infty}\right|  = 0 $ \hfill (Evaluate at infinity and simplify)\\
		$ 0 < 1 $ Therefore by the root test, the series is \fbox{\textbf{Convergent}}\\
\end {enumerate}


\fi


%========================================================================================================================

\een 


\end{document}
